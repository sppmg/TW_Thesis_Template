\documentclass[class=NCU_thesis, crop=false, float=true]{standalone}

\begin{document}
% 這裡的設定是須要在 \begin{document} 之內才能作用的
\IfStandalone{\standaloneconfig{float=false}}{} % set option for standaloneclass(sub-file)
% for sub-tex, 
% ``float=false'' disable floating environment when build sub-tex.

\fontsize{14}{25}\selectfont  % 可調間距以便閱讀 \fontsize{size}{skip} 15/30 約32每行,兩倍高
\setlength{\parindent}{2em} %縮排2字寬



% A example for define unit command.
\providecommand{\temp}[1]{$\SI{#1}{\degreeCelsius}$}

\providecommand{\symumm}[1]{$\SI{#1}{\milli\metre}$}
\providecommand{\symums}[1]{$\SI{#1}{\milli\second}$}
\providecommand{\symumM}[1]{$\SI{#1}{\milli\Molar}$}
\providecommand{\symumv}[1]{$\SI{#1}{\milli\volt}$}
\providecommand{\symuma}[1]{$\SI{#1}{\milli\ampere}$}

\providecommand{\abs}[1]{\lvert #1 \rvert } % Required package: amsmath

%%%%%%%% for codes %%%%%%%%%
\definecolor{codegreen}{rgb}{0,0.6,0}
\definecolor{codegray}{rgb}{0.5,0.5,0.5}
\definecolor{codepurple}{rgb}{0.58,0,0.82}
\definecolor{codebgcolor}{rgb}{0.95,0.95,0.92}


\lstdefinestyle{commonStyle}{
    frame=single,
    %backgroundcolor=\color{codebgcolor},
    commentstyle=\color{codegreen},
    keywordstyle=\color{blue},
    numberstyle=\small\color{codegray},
    stringstyle=\color{codepurple},
    basicstyle=\ttfamily\small,
    breakatwhitespace=false,         
    breaklines=true,                 
    captionpos=t,
    caption={\protect\filename@parse{\lstname}\protect\filename@base\text{.}\protect‌​\filename@ext}, 
        % http://tex.stackexchange.com/questions/174541/only-get-filename-and-extension-of-listing-not-whole-path
    keepspaces=true,
    xleftmargin=1cm,
    %xrightmargin=1cm,
    numbers=left,
    numbersep=5pt,
    showspaces=false,                
    showstringspaces=false,
    showtabs=false,                  
    tabsize=4
}

% for console, no line numbers
\lstdefinestyle{consoleStyle}{
    frame=single,
    %backgroundcolor=\color{codebgcolor},   
    commentstyle=\color{codegreen},
    keywordstyle=\color{blue},
    numberstyle=\small\color{codegray},
    stringstyle=\color{codepurple},
    basicstyle=\ttfamily\small,
    breakatwhitespace=false,         
    breaklines=true,                 
    captionpos=t,
    %caption={\protect\filename@parse{\lstname}\protect\filename@base\text{.}\protect‌​\filename@ext}, 
        % http://tex.stackexchange.com/questions/174541/only-get-filename-and-extension-of-listing-not-whole-path
    keepspaces=true,
    xleftmargin=0cm,
    numbers=none,
    showspaces=false,                
    showstringspaces=false,
    showtabs=false,
    tabsize=4
}

\lstdefinestyle{LatexStyle}{
    language={[LaTeX]TeX},
    inputpath={./}, % must same as root tex
    frame=single,
    %backgroundcolor=\color{backcolour},   
    commentstyle=\color{codegreen},
    keywordstyle=\color{blue},
    numberstyle=\small\color{codegray},
    stringstyle=\color{codepurple},
    basicstyle=\ttfamily\small,
    breakatwhitespace=false,         
    breaklines=true,                 
    captionpos=t,
    %caption={\protect\filename@parse{\lstname}\protect\filename@base\text{.}\protect‌​\filename@ext}, 
        % http://tex.stackexchange.com/questions/174541/only-get-filename-and-extension-of-listing-not-whole-path
    keepspaces=true,
    xleftmargin=1cm,
    xrightmargin=1cm,
    numbers=left,
    numbersep=5pt,
    showspaces=false,                
    showstringspaces=false,
    showtabs=false,                  
    tabsize=4
}

\lstset{style=commonStyle} % default style
\renewcommand{\lstlistingname}{Code} % change title of caption to ``Code'' from ``Listing'' 


\chapter{章名(章節示例)}
章內容
\section{節名}
節內容
\subsection{小節名}
 \subsubsection{小小節}
 \paragraph{段}


\chapter{文字}
第一行。
仍是第一行。 \\
第二行。


\chapter{圖片}
\subsection{插入單一圖片}
\insfig[0.15][fig:labal_test][!hbt]{logo-Linux.png}[caption][short caption]

\subsection{插入多張圖片}
\begin{figure}[!hbt]
    %\captionsetup[subfigure]{labelformat=empty} % 完全隱藏圖號
    \centering
    \subcaptionbox
        {Linux (A kernel of OS)
        \label{fig:subfig_linux}}
        {\includegraphics[width=0.3\linewidth]{logo-Linux.png}}
    ~
    \subcaptionbox
        {Debain (A popular distribution). Demo long caption,  bla bla bla bla bla.
        \label{fig:subfig_debian}}
        {\includegraphics[width=0.3\linewidth]{logo-Debian.png}}
    \vspace{\baselineskip} % 分隔上下列
    \subcaptionbox
        {GNU (A project of OS)
        \label{fig:subfig_gnu}}
        {\includegraphics[width=0.6\linewidth]{logo-GNU.png}}
     % use ``\subref{fig:subfig_debian}'' get ID of subfigure(this ID is Debian)
    \caption{caption, 使用 \subref{fig:subfig_debian}取得子圖(Debian)編號 }
    \label{fig:labal}
\end{figure}


\chapter{表格}
\subsection{一般表格}
\begin{table}[h]
    \centering
    \caption{Solution}
    \begin{tabular}{| l | l |}
        \hline
        Component & Concentration(mM) \\ \hline
        \ce{NaCl} & 118.0 \\ \hline
    \end{tabular}
\end{table}

\subsection{自動折行表格}
\begin{table}[h]
    \centering
    \begin{tabularx}{\textwidth}{| l | X |}
        \hline
        short & short short \\ \hline
        long & long long long long long long long long long long \\ \hline
    \end{tabularx}
\end{table}

\end{document}