%%%%% build setting | 編譯設定 %%%%%
\provideboolean{publish} % publish ? Set true before publish. 發怖前設true
\setboolean{publish}{false} % {true}/{false}

% draft option add to \documentclass[draft]

\synctex=1 % 啟用SyncTeX

%%%%% Information of your document. | 定義文件資訊 %%%%%
\def\deptshort {}
\def\dept      {}%XX學系XXX碩士班(請參考「中央大學學位論文撰寫體例參考」附錄)
\def\degree    {}
\def\title     {\LaTeX\ 論文樣板與簡易教學}
\def\subtitle  {\LaTeX\ Thesis Template and Tutorial}
\def\logo      {}   % 填入校徽檔名。中央無校徽在封面,維持空白可除去校徽
\def\author    {sppmg}
\def\mprof     {}
\def\sprofi    {} % 共同指導 1
\def\sprofii   {}      % 共同指導 2                  
\def\degreedate{中~華~民~國~一百零五~年~十~月}
\def\copyyear  {2012-2013}
\provideboolean{printcopyright} % print copyright text on titlepage or cover.
\setboolean{printcopyright}{false} % {true}/{false}

\def\keywordsZh{latex, 中文, 論文, sppmg} % don't use \def\keywords
\def\keywordsEn{latex, chinese, thesis, sppmg}

% \addbibresource{introduction.bib} % 填上你的文獻資料庫(Biblatex 格式),以多次使用以加入多個來源。

%%%%%% set OS | 設定作業系統 %%%%%
\def\OS{linux} % {linux}/{win}, only effect auto select CJK font.(CJK means Chinese, Japanese, and Korean)

%%%%%% set font | 設定中英文字型 %%%%%
% keep empty for default font. CJK font must set OS for auto select.
% Linux 利用指令 fc-list :lang=zh 來查詢可以用的字體名稱。
\def\mainfont   {}
\def\sansfont   {}
\def\monofont   {} %DejaVu Sans Mono
\def\CJKmainfont{}
\def\CJKsansfont{}
\def\CJKmonofont{}

%%%%% depth | 章節深度 %%%%%
% LaTeX 預設2,chapter == 0。
\setcounter{secnumdepth}{4} % 設定章節標題給予數字標號的深度, \paragraph == 4。
\setcounter{tocdepth}{2}  % 目錄顯示層級,\subsection == 2。

%%%%% style of toc and titles | 目錄及章節風格 %%%%%
% see tutorial.
\def\tocStyle{0}
\def\titleStyle{0}
    \def\indentblocksss{0mm}    % indent \subsubsection{}
    \def\indentblockpar{0mm}    % indent \paragraph{}